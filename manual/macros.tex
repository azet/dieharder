% This file is filled in by ../configure.  Do not edit this file by
% hand!  Make changes in macros.tex.in instead.
%
%                      Macros for dieharder.tex
%
% These macros basicallypermit one to shorten typing -- it is much easier
% to type \be than \begin{equation}.  Others encapsulate complex commands,
% such as commands that add dynamic links to wikipedia content (which is
% VERY USEFUL for online content!  I'm quite proud of my "\wikinote{}"
% below, as it is nearly a one-stop-shop for references these days.
%

\newcommand{\version}{2.24.2}

% Yes, the manual (or the derived book) may one day need some math
% macros, as there is some HEAVY math underlying many of the tests.
\newcommand{\Vec}[1]{\mbox{\boldmath $#1$}}
\newcommand{\Hat}[1]{\hat{\mbox{\boldmath $#1$}}}
\newcommand{\partialdiv}[2]{\frac{\partial #1}{\partial #2}}
\newcommand{\ppartialdiv}[2]{\frac{\partial^2 #1}{\partial #2^2}}
\newcommand{\del}{{\bf \Vec{\nabla}}}
\newcommand{\deldot}{\del \cdot}
\newcommand{\curl}{\del \times}
\newcommand{\lapl}{\nabla^2}
\newcommand{\vsh}[2]{\Vec{Y}_{#1}^{#2}}
\newcommand{\abs}[1]{\left| #1 \right|}
\newcommand{\RE}{\rm I \hspace{-.180em} R}
\newcommand{\be}{ \begin{equation} }
\newcommand{\ee}{ \end{equation} }
\newcommand{\bea}{ \begin{eqnarray} }
\newcommand{\eea}{ \end{eqnarray} }
\newcommand{\bi}{ \begin{itemize} }
\newcommand{\ei}{ \end{itemize} }
\newcommand{\ben}{ \begin{enumerate} }
\newcommand{\een}{ \end{enumerate} }
\newcommand{\bv}{ \begin{verbatim} }
% one cannot end verbatim with a macro, as the macro
% is interpreted as verbatim text, duh...
\newcommand{\pagefill}{\vspace*{\fill} \newpage \vspace*{\fill}}
\renewcommand{\deg}{^{\circ}}

% Clever link macros...
\usepackage{url}

\newcommand{\link}[2]{\url{#1}{#2}}
\newcommand{\GIYF}{\url{GIYF}{http://www.google.com}}
\newcommand{\WIYF}{\url{WIYF}{http://www.wikipedia.org}}
\newcommand{\MWIYF}{\url{MWIYF}{http://mathworld.wolfram.com}}
\newcommand{\wikinote}[2]{\footnote{Wikipedia: \url{http://www.wikipedia.org/wiki/#1.}{http://www.wikipedia.org/wiki/#1} #2}}
\newcommand{\googlenote}[2]{\footnote{GIYF: \url{#1 }{http://www.google.com/search?hl=en&lr=&q=#1&btnG=Search} #2}}
